
\documentclass[conference]{IEEEtran}
\IEEEoverridecommandlockouts



% *** CITATION PACKAGES ***
%
\usepackage{cite}
% cite.sty was written by Donald Arseneau
% V1.6 and later of IEEEtran pre-defines the format of the cite.sty package
% \cite{} output to follow that of IEEE. Loading the cite package will
% result in citation numbers being automatically sorted and properly
% "compressed/ranged". e.g., [1], [9], [2], [7], [5], [6] without using
% cite.sty will become [1], [2], [5]--[7], [9] using cite.sty. cite.sty's
% \cite will automatically add leading space, if needed. Use cite.sty's
% noadjust option (cite.sty V3.8 and later) if you want to turn this off
% such as if a citation ever needs to be enclosed in parenthesis.
% cite.sty is already installed on most LaTeX systems. Be sure and use
% version 5.0 (2009-03-20) and later if using hyperref.sty.
% The latest version can be obtained at:
% http://www.ctan.org/tex-archive/macros/latex/contrib/cite/
% The documentation is contained in the cite.sty file itself.






% *** GRAPHICS RELATED PACKAGES ***
%
\ifCLASSINFOpdf
  % \usepackage[pdftex]{graphicx}
  % declare the path(s) where your graphic files are
  % \graphicspath{{../pdf/}{../jpeg/}}
  % and their extensions so you won't have to specify these with
  % every instance of \includegraphics
  % \DeclareGraphicsExtensions{.pdf,.jpeg,.png}
\else
  % or other class option (dvipsone, dvipdf, if not using dvips). graphicx
  % will default to the driver specified in the system graphics.cfg if no
  % driver is specified.
  % \usepackage[dvips]{graphicx}
  % declare the path(s) where your graphic files are
  % \graphicspath{{../eps/}}
  % and their extensions so you won't have to specify these with
  % every instance of \includegraphics
  % \DeclareGraphicsExtensions{.eps}
\fi
% graphicx was written by David Carlisle and Sebastian Rahtz. It is
% required if you want graphics, photos, etc. graphicx.sty is already
% installed on most LaTeX systems. The latest version and documentation
% can be obtained at: 
% http://www.ctan.org/tex-archive/macros/latex/required/graphics/
% Another good source of documentation is "Using Imported Graphics in
% LaTeX2e" by Keith Reckdahl which can be found at:
% http://www.ctan.org/tex-archive/info/epslatex/
%
% latex, and pdflatex in dvi mode, support graphics in encapsulated
% postscript (.eps) format. pdflatex in pdf mode supports graphics
% in .pdf, .jpeg, .png and .mps (metapost) formats. Users should ensure
% that all non-photo figures use a vector format (.eps, .pdf, .mps) and
% not a bitmapped formats (.jpeg, .png). IEEE frowns on bitmapped formats
% which can result in "jaggedy"/blurry rendering of lines and letters as
% well as large increases in file sizes.
%
% You can find documentation about the pdfTeX application at:
% http://www.tug.org/applications/pdftex





% *** MATH PACKAGES ***
%
%\usepackage[cmex10]{amsmath}
% A popular package from the American Mathematical Society that provides
% many useful and powerful commands for dealing with mathematics. If using
% it, be sure to load this package with the cmex10 option to ensure that
% only type 1 fonts will utilized at all point sizes. Without this option,
% it is possible that some math symbols, particularly those within
% footnotes, will be rendered in bitmap form which will result in a
% document that can not be IEEE Xplore compliant!
%
% Also, note that the amsmath package sets \interdisplaylinepenalty to 10000
% thus preventing page breaks from occurring within multiline equations. Use:
%\interdisplaylinepenalty=2500
% after loading amsmath to restore such page breaks as IEEEtran.cls normally
% does. amsmath.sty is already installed on most LaTeX systems. The latest
% version and documentation can be obtained at:
% http://www.ctan.org/tex-archive/macros/latex/required/amslatex/math/





% *** SPECIALIZED LIST PACKAGES ***
%
%\usepackage{algorithmic}
% algorithmic.sty was written by Peter Williams and Rogerio Brito.
% This package provides an algorithmic environment fo describing algorithms.
% You can use the algorithmic environment in-text or within a figure
% environment to provide for a floating algorithm. Do NOT use the algorithm
% floating environment provided by algorithm.sty (by the same authors) or
% algorithm2e.sty (by Christophe Fiorio) as IEEE does not use dedicated
% algorithm float types and packages that provide these will not provide
% correct IEEE style captions. The latest version and documentation of
% algorithmic.sty can be obtained at:
% http://www.ctan.org/tex-archive/macros/latex/contrib/algorithms/
% There is also a support site at:
% http://algorithms.berlios.de/index.html
% Also of interest may be the (relatively newer and more customizable)
% algorithmicx.sty package by Szasz Janos:
% http://www.ctan.org/tex-archive/macros/latex/contrib/algorithmicx/




% *** ALIGNMENT PACKAGES ***
%
%\usepackage{array}
% Frank Mittelbach's and David Carlisle's array.sty patches and improves
% the standard LaTeX2e array and tabular environments to provide better
% appearance and additional user controls. As the default LaTeX2e table
% generation code is lacking to the point of almost being broken with
% respect to the quality of the end results, all users are strongly
% advised to use an enhanced (at the very least that provided by array.sty)
% set of table tools. array.sty is already installed on most systems. The
% latest version and documentation can be obtained at:
% http://www.ctan.org/tex-archive/macros/latex/required/tools/


% IEEEtran contains the IEEEeqnarray family of commands that can be used to
% generate multiline equations as well as matrices, tables, etc., of high
% quality.




% *** SUBFIGURE PACKAGES ***
%\ifCLASSOPTIONcompsoc
%  \usepackage[caption=false,font=normalsize,labelfont=sf,textfont=sf]{subfig}
%\else
%  \usepackage[caption=false,font=footnotesize]{subfig}
%\fi
% subfig.sty, written by Steven Douglas Cochran, is the modern replacement
% for subfigure.sty, the latter of which is no longer maintained and is
% incompatible with some LaTeX packages including fixltx2e. However,
% subfig.sty requires and automatically loads Axel Sommerfeldt's caption.sty
% which will override IEEEtran.cls' handling of captions and this will result
% in non-IEEE style figure/table captions. To prevent this problem, be sure
% and invoke subfig.sty's "caption=false" package option (available since
% subfig.sty version 1.3, 2005/06/28) as this is will preserve IEEEtran.cls
% handling of captions.
% Note that the Computer Society format requires a larger sans serif font
% than the serif footnote size font used in traditional IEEE formatting
% and thus the need to invoke different subfig.sty package options depending
% on whether compsoc mode has been enabled.
%
% The latest version and documentation of subfig.sty can be obtained at:
% http://www.ctan.org/tex-archive/macros/latex/contrib/subfig/




% *** FLOAT PACKAGES ***
%
%\usepackage{fixltx2e}
% fixltx2e, the successor to the earlier fix2col.sty, was written by
% Frank Mittelbach and David Carlisle. This package corrects a few problems
% in the LaTeX2e kernel, the most notable of which is that in current
% LaTeX2e releases, the ordering of single and double column floats is not
% guaranteed to be preserved. Thus, an unpatched LaTeX2e can allow a
% single column figure to be placed prior to an earlier double column
% figure. The latest version and documentation can be found at:
% http://www.ctan.org/tex-archive/macros/latex/base/


%\usepackage{stfloats}
% stfloats.sty was written by Sigitas Tolusis. This package gives LaTeX2e
% the ability to do double column floats at the bottom of the page as well
% as the top. (e.g., "\begin{figure*}[!b]" is not normally possible in
% LaTeX2e). It also provides a command:
%\fnbelowfloat
% to enable the placement of footnotes below bottom floats (the standard
% LaTeX2e kernel puts them above bottom floats). This is an invasive package
% which rewrites many portions of the LaTeX2e float routines. It may not work
% with other packages that modify the LaTeX2e float routines. The latest
% version and documentation can be obtained at:
% http://www.ctan.org/tex-archive/macros/latex/contrib/sttools/
% Do not use the stfloats baselinefloat ability as IEEE does not allow
% \baselineskip to stretch. Authors submitting work to the IEEE should note
% that IEEE rarely uses double column equations and that authors should try
% to avoid such use. Do not be tempted to use the cuted.sty or midfloat.sty
% packages (also by Sigitas Tolusis) as IEEE does not format its papers in
% such ways.
% Do not attempt to use stfloats with fixltx2e as they are incompatible.
% Instead, use Morten Hogholm'a dblfloatfix which combines the features
% of both fixltx2e and stfloats:
%
% \usepackage{dblfloatfix}
% The latest version can be found at:
% http://www.ctan.org/tex-archive/macros/latex/contrib/dblfloatfix/




% *** PDF, URL AND HYPERLINK PACKAGES ***
%
%\usepackage{url}
% url.sty was written by Donald Arseneau. It provides better support for
% handling and breaking URLs. url.sty is already installed on most LaTeX
% systems. The latest version and documentation can be obtained at:
% http://www.ctan.org/tex-archive/macros/latex/contrib/url/
% Basically, \url{my_url_here}.

\usepackage{tikz}
\usetikzlibrary{shapes,arrows}
\usepackage{pgfplots}


% *** Do not adjust lengths that control margins, column widths, etc. ***
% *** Do not use packages that alter fonts (such as pslatex).         ***
% There should be no need to do such things with IEEEtran.cls V1.6 and later.
% (Unless specifically asked to do so by the journal or conference you plan
% to submit to, of course. )


% correct bad hyphenation here
\hyphenation{op-tical net-works semi-conduc-tor}


\begin{document}
%
% paper title
% Titles are generally capitalized except for words such as a, an, and, as,
% at, but, by, for, in, nor, of, on, or, the, to and up, which are usually
% not capitalized unless they are the first or last word of the title.
% Linebreaks \\ can be used within to get better formatting as desired.
% Do not put math or special symbols in the title.
\title{One Solution of the Accurate Summation \\ Using Fixed-point Accumulator}


\author{
	\thanks{
		This work was partially supported by the Ministry of Education, 
		Science and Technological Development of the Republic of Serbia, 
		under grant number: TR32029.
	}
	Jelena Jankovic,\thanks{
		Jelena Jankovic is with the University of Novi Sad, Faculty of Technical sciences, 
		Computing and control engineering dept., 
		Trg Dositeja Obradovica 6, 21000 Novi Sad, Serbia
		(phone: 381-21-6623169; e-mail: jelenaa31051994@gmail.com)
	}
	Milos Subotic,\thanks{
		Milos Subotic is with RT-RK Institute for Computer Based Systems,
		Narodnog fronta 23a, 21000 Novi Sad, Serbia
		(phone: 381-21-4801291; e-mail: milos.subotic@rt-rk.com)
	}
	Vladimir Marinkovic\thanks{
		Vladimir Marinkovic is with RT-RK Institute for Computer based systems,
		Narodnog fronta 23a, 21000 Novi Sad, Serbia
		(phone: 381-21-4801291; e-mail: vladimir.marinkovic@rt-rk.com)
	}
}


\maketitle


\begin{abstract}
The accurate summation is algorithm for summing floating-point numbers with reduced rounding errors.
This paper presents one solution of accurate summation problem using fixed-point accumulator.
Implemented algorithm is compared with others
and has performances and precision comparable with the best existing algorithms.
\end{abstract}

\begin{IEEEkeywords}
Accurate summation, summation algorithm, floating-point numbers, fixed-point, rounding errors
\end{IEEEkeywords}




% For peer review papers, you can put extra information on the cover
% page as needed:
% \ifCLASSOPTIONpeerreview
% \begin{center} \bfseries EDICS Category: 3-BBND \end{center}
% \fi
%
% For peerreview papers, this IEEEtran command inserts a page break and
% creates the second title. It will be ignored for other modes.
\IEEEpeerreviewmaketitle


\section{Introduction}
% no \IEEEPARstart
An accurate summation \cite{Higham} is algorithm for summing 
floating-point numbers \cite{WhatShouldKnowAboutFP}. It is well known that the summation 
of large sets of numbers can be very inaccurate due
to the accumulation of rounding errors. 
\par
The typical problem is 'one large, many smalls' \cite{ComparisonOfMethods} 
where one big number is being summed with many small numbers. 
For example, in four-digit decade floating-point space,
if 1000 small numbers 0.001 is accumulated to one large 
number 1.000, result would be 2.000. On the other hand,
if one large number 1.000 is summed with 1000 small numbers 0.001,
result would be 1.000 instead of 2.000 because of rounding errors.
\par
Those performances were measured, the accuracy was verified and this 
solution has been compared with other algorithms which is 
shown in further captions. This paper describes novel accurate 
summation algorithm utilizing fixed-point accumulator.
Algorithm is implemented in C++.


\section{Related work}
Several authors have used variety of methods for floating-point 
summation to find out which methods achieve the best accuracy. 
\par
The first algorithm is Kahan's method \cite{ComparisonOfMethods}. 
His compensated summation method employs the correction 
on every step of a recursive summation. After each partial 
sum is formed, the correction is computed and immediately added 
to the next term Xi before that term is added to the partial
sum. Thus the idea is to capture the rounding errors and feed 
them back into the summation. The method maybe written as follow
\par 
This paper shows comparison of few methods not all of them.
In \cite{ComparisonOfMethods, CascadingAccumulators} many algorithms 
are compared with many problems. The best of them is the method of Cascading 
accumulators. Algorithm described in this paper has been compared to the method of Cascading 
accumulators, because the method of Cascading accumulators is proved 
to be the best in performance and precision in \cite{ComparisoOfMethods}.
The fixed-point accumulator algorithm has no errors as well as the method of Cascading 
accumulators.
\par
There are also a few more algorithms. 
The most important is Kahan-Babuska-Klein summation algorithm \cite{KahanBabuskaKlein}.
The solution written on this paper has outperforming precision and performances.


\section{Implementation}
The fixed-point accumulator is implemented as c++ number class.
Also, it consists of five 64-bit numbers. The compiler puts these 
numbers into registers, so numbers do not go into cash
memory and therefore this solution is faster than another ones.
So far, we have made only single precision fixed-point 
accumulator. However, double precision accumulator could also 
be done without any problems. The inline methods are used to give compiler
the opportunity to make a good optimization and to fit
everything into registers.


\section{Measurements}
This paper presents three problems: 'many smalls, one large',
'one large, many smalls' and 'one large, many smalls, many tinies'.
In the following, fixed-point algorithm is compared with Kahan's method
which is classic algorithm and the method of Cascading accumulators 
which presents the best solution due to its performance and precision 
proved in paper \cite{CascadingAccumulators}.


\par
'Many smalls, one large' is problem where many small numbers 
are being summed with one large number. If $2^24 *\epsilon$ small 
numbers are accumulated to one large number 1.0, where $\epsilon$ 
is defined as $b^{-(p-1)}/2$ by Demmel, the result would be 2.0. 
\par 
'One large, many smalls' is problem where one large number 
is being summed with many small numbers. It means that
if one large number 1.0 is summed with small 
numbers $2^24 *\epsilon$, where $\epsilon$ is defined 
as $b^{-(p-1)}/2$ by Demmel, the result would be 1.0 instead 
of 2.0 because of rounding errors.
\par 
'One large, many smalls, many tinies' is problem where one 
large number is being summed with $2^24$ small numbers and $2^26$
tiny numbers. One large number 1.0 is 
summed with small numbers $2^24 *\epsilon$ and tiny 
numbers $2^26 *\epsilon^2$, where $\epsilon$ is defined as $b^{-(p-1)}/2$ by Demmel.
\par
The various methods and data were tested by running C++ programs.
The results are shown in Table below. For each method and data 
set we show two numbers, one that presents the time required for 
execution and another that presents obtained value of the algorithm.

\begin{table}[!t]
\renewcommand{\arraystretch}{1.3}
\caption{Result and measured time for 100 iteration}
\label{tab:measurement}
\begin{tabular}{|c|c|c|c|}
%TODO Designate row and column with method and problem.
\hline
	\parbox[]{1.8cm}{ \center } & 
	\parbox[]{1.7cm}{ \centering Many smalls, one large } & 
	\parbox[]{1.7cm}{ \center One large, many smalls } & 
	\parbox[]{1.7cm}{ \center One large, many smalls, many tinies \linebreak } \\
\hline
	Correct result &
	\parbox[]{1.7cm}{ \center 2.0 \linebreak} &
	\parbox[]{1.7cm}{ \center 2.0 \linebreak} &
	\parbox[]{1.7cm}{ \center $2.0 + 4\epsilon$ \linebreak} \\
\hline
	Ordinary sum &
	\parbox[]{1.7cm}{ \center 2.0 \linebreak 7.194s \linebreak} &
	\parbox[]{1.7cm}{ \center 1.0 \linebreak 7.258s \linebreak} &
	\parbox[]{1.7cm}{ \center 1.0 \linebreak 34.30s \linebreak} \\
\hline
	Kahan &
	\parbox[]{1.7cm}{ \center 2.0 \linebreak 12.70s \linebreak} &
	\parbox[]{1.7cm}{ \center 2.0 \linebreak 12.69s \linebreak} &
	\parbox[]{1.7cm}{ \center 2.0 \linebreak 62.32s \linebreak} \\
\hline
	\parbox[]{1.7cm}{ \center Cascade \linebreak accumulator \linebreak} &
	\parbox[]{1.7cm}{ \center 2.0 \linebreak 7.008s \linebreak} &
	\parbox[]{1.7cm}{ \center 2.0 \linebreak 6.977s \linebreak} &
	\parbox[]{1.7cm}{ \center $2.0 + 4\epsilon$ \linebreak 34.93s \linebreak} \\
\hline
	\parbox[]{1.7cm}{ \center Fixed-point \linebreak accumulator \linebreak} &
	\parbox[]{1.7cm}{ \center 2.0 \linebreak 11.69s \linebreak} &
	\parbox[]{1.7cm}{ \center 2.0 \linebreak 11.70s \linebreak} &
	\parbox[]{1.7cm}{ \center $2.0 + 4\epsilon$ \linebreak 58.53s \linebreak} \\	
\hline
\end{tabular}
\end{table}

\par  
This paper presents solution that is comparable with Cascading 
accumulator in performances. Reason for that is that those 64 
double accumulators are in caches memory and fixed-point accumulator 
has a bit more work that is a bit larger number of instructions 
per summing than the Cascading accumulator, so it comes to the same.
\par 
The solution is not half as fast as Cascading accumulator but it 
occupies a significantly smaller memory space. More than 12 times 
less memory is needed for fixed-point accumulator than for Cascading accumulator.
\par 
Kahan's algorithm does not work for problem called 'one large, many smalls, many tinies'
because it needs recursive Kahan's algorithm with variable c which is used for storing rounding errors. 
For example, according to previous explanation of problem 'one large, many smalls, many tines', 
it is expected to get a result $2.0 + 4\epsilon$, where $\epsilon$ is defined as $b^{-(p-1)}/2$ by Demmel, 
but the result is 2.000. The reason for this error lies in need of 
recursive Kahan inside regular Kahan's algorithm. This method has 
comparable performances with accurate summation using fixed-point accumulator.
\par 
We conclude that the Cascading Accumulator method is not the
best algorithm.It needs double precision fixed-point accumulators.
These accumulators are not always available. Due to this facts, implementation of 
Cascading Accumulators for double precision is not recommended. 
This paper shows that fixed-point accumulator have no problems of 
using double precision while some architectures do not have greater float type of double. 
\par 
Fixed-point accumulator would be better in hardware due to the fact that it is less logic occupied.

\section{Conclusion}
According to the above table it has been found and proved
that the accurate summation using fixed-point accumulator 
algorithm is better than the method of Cascading Accumulators 
as well as Kahan's algorithm and all the others.
\par
Further improvement can be achieved by optimizing this
solution in assembler and implementing double precision version in hardware. 
This solution should also be compared to some more 
algorithms and problems. 

% use section* for acknowledgment
\section*{Acknowledgment}
This work was partially supported by the Ministry of Education, 
Science and Technological Development of the Republic of Serbia, 
under grant number: TR32029.


\bibliographystyle{IEEEtran}
\bibliography{cites}


\end{document}
