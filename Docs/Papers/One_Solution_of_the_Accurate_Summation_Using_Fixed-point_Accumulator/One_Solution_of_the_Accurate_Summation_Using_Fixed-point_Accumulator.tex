
\documentclass[conference]{IEEEtran}
\IEEEoverridecommandlockouts



% *** CITATION PACKAGES ***
%
\usepackage{cite}
% cite.sty was written by Donald Arseneau
% V1.6 and later of IEEEtran pre-defines the format of the cite.sty package
% \cite{} output to follow that of IEEE. Loading the cite package will
% result in citation numbers being automatically sorted and properly
% "compressed/ranged". e.g., [1], [9], [2], [7], [5], [6] without using
% cite.sty will become [1], [2], [5]--[7], [9] using cite.sty. cite.sty's
% \cite will automatically add leading space, if needed. Use cite.sty's
% noadjust option (cite.sty V3.8 and later) if you want to turn this off
% such as if a citation ever needs to be enclosed in parenthesis.
% cite.sty is already installed on most LaTeX systems. Be sure and use
% version 5.0 (2009-03-20) and later if using hyperref.sty.
% The latest version can be obtained at:
% http://www.ctan.org/tex-archive/macros/latex/contrib/cite/
% The documentation is contained in the cite.sty file itself.






% *** GRAPHICS RELATED PACKAGES ***
%
\ifCLASSINFOpdf
  % \usepackage[pdftex]{graphicx}
  % declare the path(s) where your graphic files are
  % \graphicspath{{../pdf/}{../jpeg/}}
  % and their extensions so you won't have to specify these with
  % every instance of \includegraphics
  % \DeclareGraphicsExtensions{.pdf,.jpeg,.png}
\else
  % or other class option (dvipsone, dvipdf, if not using dvips). graphicx
  % will default to the driver specified in the system graphics.cfg if no
  % driver is specified.
  % \usepackage[dvips]{graphicx}
  % declare the path(s) where your graphic files are
  % \graphicspath{{../eps/}}
  % and their extensions so you won't have to specify these with
  % every instance of \includegraphics
  % \DeclareGraphicsExtensions{.eps}
\fi
% graphicx was written by David Carlisle and Sebastian Rahtz. It is
% required if you want graphics, photos, etc. graphicx.sty is already
% installed on most LaTeX systems. The latest version and documentation
% can be obtained at: 
% http://www.ctan.org/tex-archive/macros/latex/required/graphics/
% Another good source of documentation is "Using Imported Graphics in
% LaTeX2e" by Keith Reckdahl which can be found at:
% http://www.ctan.org/tex-archive/info/epslatex/
%
% latex, and pdflatex in dvi mode, support graphics in encapsulated
% postscript (.eps) format. pdflatex in pdf mode supports graphics
% in .pdf, .jpeg, .png and .mps (metapost) formats. Users should ensure
% that all non-photo figures use a vector format (.eps, .pdf, .mps) and
% not a bitmapped formats (.jpeg, .png). IEEE frowns on bitmapped formats
% which can result in "jaggedy"/blurry rendering of lines and letters as
% well as large increases in file sizes.
%
% You can find documentation about the pdfTeX application at:
% http://www.tug.org/applications/pdftex





% *** MATH PACKAGES ***
%
%\usepackage[cmex10]{amsmath}
% A popular package from the American Mathematical Society that provides
% many useful and powerful commands for dealing with mathematics. If using
% it, be sure to load this package with the cmex10 option to ensure that
% only type 1 fonts will utilized at all point sizes. Without this option,
% it is possible that some math symbols, particularly those within
% footnotes, will be rendered in bitmap form which will result in a
% document that can not be IEEE Xplore compliant!
%
% Also, note that the amsmath package sets \interdisplaylinepenalty to 10000
% thus preventing page breaks from occurring within multiline equations. Use:
%\interdisplaylinepenalty=2500
% after loading amsmath to restore such page breaks as IEEEtran.cls normally
% does. amsmath.sty is already installed on most LaTeX systems. The latest
% version and documentation can be obtained at:
% http://www.ctan.org/tex-archive/macros/latex/required/amslatex/math/





% *** SPECIALIZED LIST PACKAGES ***
%
%\usepackage{algorithmic}
% algorithmic.sty was written by Peter Williams and Rogerio Brito.
% This package provides an algorithmic environment fo describing algorithms.
% You can use the algorithmic environment in-text or within a figure
% environment to provide for a floating algorithm. Do NOT use the algorithm
% floating environment provided by algorithm.sty (by the same authors) or
% algorithm2e.sty (by Christophe Fiorio) as IEEE does not use dedicated
% algorithm float types and packages that provide these will not provide
% correct IEEE style captions. The latest version and documentation of
% algorithmic.sty can be obtained at:
% http://www.ctan.org/tex-archive/macros/latex/contrib/algorithms/
% There is also a support site at:
% http://algorithms.berlios.de/index.html
% Also of interest may be the (relatively newer and more customizable)
% algorithmicx.sty package by Szasz Janos:
% http://www.ctan.org/tex-archive/macros/latex/contrib/algorithmicx/




% *** ALIGNMENT PACKAGES ***
%
%\usepackage{array}
% Frank Mittelbach's and David Carlisle's array.sty patches and improves
% the standard LaTeX2e array and tabular environments to provide better
% appearance and additional user controls. As the default LaTeX2e table
% generation code is lacking to the point of almost being broken with
% respect to the quality of the end results, all users are strongly
% advised to use an enhanced (at the very least that provided by array.sty)
% set of table tools. array.sty is already installed on most systems. The
% latest version and documentation can be obtained at:
% http://www.ctan.org/tex-archive/macros/latex/required/tools/


% IEEEtran contains the IEEEeqnarray family of commands that can be used to
% generate multiline equations as well as matrices, tables, etc., of high
% quality.




% *** SUBFIGURE PACKAGES ***
%\ifCLASSOPTIONcompsoc
%  \usepackage[caption=false,font=normalsize,labelfont=sf,textfont=sf]{subfig}
%\else
%  \usepackage[caption=false,font=footnotesize]{subfig}
%\fi
% subfig.sty, written by Steven Douglas Cochran, is the modern replacement
% for subfigure.sty, the latter of which is no longer maintained and is
% incompatible with some LaTeX packages including fixltx2e. However,
% subfig.sty requires and automatically loads Axel Sommerfeldt's caption.sty
% which will override IEEEtran.cls' handling of captions and this will result
% in non-IEEE style figure/table captions. To prevent this problem, be sure
% and invoke subfig.sty's "caption=false" package option (available since
% subfig.sty version 1.3, 2005/06/28) as this is will preserve IEEEtran.cls
% handling of captions.
% Note that the Computer Society format requires a larger sans serif font
% than the serif footnote size font used in traditional IEEE formatting
% and thus the need to invoke different subfig.sty package options depending
% on whether compsoc mode has been enabled.
%
% The latest version and documentation of subfig.sty can be obtained at:
% http://www.ctan.org/tex-archive/macros/latex/contrib/subfig/




% *** FLOAT PACKAGES ***
%
%\usepackage{fixltx2e}
% fixltx2e, the successor to the earlier fix2col.sty, was written by
% Frank Mittelbach and David Carlisle. This package corrects a few problems
% in the LaTeX2e kernel, the most notable of which is that in current
% LaTeX2e releases, the ordering of single and double column floats is not
% guaranteed to be preserved. Thus, an unpatched LaTeX2e can allow a
% single column figure to be placed prior to an earlier double column
% figure. The latest version and documentation can be found at:
% http://www.ctan.org/tex-archive/macros/latex/base/


%\usepackage{stfloats}
% stfloats.sty was written by Sigitas Tolusis. This package gives LaTeX2e
% the ability to do double column floats at the bottom of the page as well
% as the top. (e.g., "\begin{figure*}[!b]" is not normally possible in
% LaTeX2e). It also provides a command:
%\fnbelowfloat
% to enable the placement of footnotes below bottom floats (the standard
% LaTeX2e kernel puts them above bottom floats). This is an invasive package
% which rewrites many portions of the LaTeX2e float routines. It may not work
% with other packages that modify the LaTeX2e float routines. The latest
% version and documentation can be obtained at:
% http://www.ctan.org/tex-archive/macros/latex/contrib/sttools/
% Do not use the stfloats baselinefloat ability as IEEE does not allow
% \baselineskip to stretch. Authors submitting work to the IEEE should note
% that IEEE rarely uses double column equations and that authors should try
% to avoid such use. Do not be tempted to use the cuted.sty or midfloat.sty
% packages (also by Sigitas Tolusis) as IEEE does not format its papers in
% such ways.
% Do not attempt to use stfloats with fixltx2e as they are incompatible.
% Instead, use Morten Hogholm'a dblfloatfix which combines the features
% of both fixltx2e and stfloats:
%
% \usepackage{dblfloatfix}
% The latest version can be found at:
% http://www.ctan.org/tex-archive/macros/latex/contrib/dblfloatfix/




% *** PDF, URL AND HYPERLINK PACKAGES ***
%
%\usepackage{url}
% url.sty was written by Donald Arseneau. It provides better support for
% handling and breaking URLs. url.sty is already installed on most LaTeX
% systems. The latest version and documentation can be obtained at:
% http://www.ctan.org/tex-archive/macros/latex/contrib/url/
% Basically, \url{my_url_here}.

\usepackage{tikz}
\usetikzlibrary{shapes,arrows}
\usepackage{pgfplots}


% *** Do not adjust lengths that control margins, column widths, etc. ***
% *** Do not use packages that alter fonts (such as pslatex).         ***
% There should be no need to do such things with IEEEtran.cls V1.6 and later.
% (Unless specifically asked to do so by the journal or conference you plan
% to submit to, of course. )


% correct bad hyphenation here
%\hyphenation{op-tical net-works semi-conduc-tor}


\begin{document}
%
% paper title
% Titles are generally capitalized except for words such as a, an, and, as,
% at, but, by, for, in, nor, of, on, or, the, to and up, which are usually
% not capitalized unless they are the first or last word of the title.
% Linebreaks \\ can be used within to get better formatting as desired.
% Do not put math or special symbols in the title.
\title{One Solution of the Accurate Summation \\ Using Fixed-point Accumulator}


\author{
	\thanks{
		This work was partially supported by the Ministry of Education, 
		Science and Technological Development of the Republic of Serbia, 
		under grant number: TR32029.
	}
	Jelena Jankovic,\thanks{
		Jelena Jankovic is with the University of Novi Sad, Faculty of Technical sciences, 
		Computing and control engineering dept., 
		Trg Dositeja Obradovica 6, 21000 Novi Sad, Serbia
		(phone: 381-21-6623169; e-mail: jelenaa31051994@gmail.com)
	}
	Milos Subotic,\thanks{
		Milos Subotic is with RT-RK Institute for Computer Based Systems,
		Narodnog fronta 23a, 21000 Novi Sad, Serbia
		(phone: 381-21-4801291; e-mail: milos.subotic@rt-rk.com)
	}
	Vladimir Marinkovic\thanks{
		Vladimir Marinkovic is with RT-RK Institute for Computer based systems,
		Narodnog fronta 23a, 21000 Novi Sad, Serbia
		(phone: 381-21-4801291; e-mail: vladimir.marinkovic@rt-rk.com)
	}
}


\maketitle


\begin{abstract}
The accurate summation is algorithm for summing floating-point numbers with reduced rounding errors.
This paper presents one solution of accurate summation problem using fixed-point accumulator.
Implemented algorithm is compared with others
and has performances and precision comparable with the best existing algorithms.
\end{abstract}

\begin{IEEEkeywords}
Accurate summation, summation algorithm, floating-point numbers, fixed-point, rounding errors
\end{IEEEkeywords}




% For peer review papers, you can put extra information on the cover
% page as needed:
% \ifCLASSOPTIONpeerreview
% \begin{center} \bfseries EDICS Category: 3-BBND \end{center}
% \fi
%
% For peerreview papers, this IEEEtran command inserts a page break and
% creates the second title. It will be ignored for other modes.
\IEEEpeerreviewmaketitle


\section{Introduction}
% no \IEEEPARstart
An accurate summation \cite{Higham} is algorithm for summing 
floating-point numbers \cite{WhatShouldKnowAboutFP}. 
It is well known that the summation 
of large sets of numbers can be very inaccurate due
to the accumulation of rounding errors. 
\par
The typical problem is 'One large, many smalls' \cite{ComparisonOfMethods} 
where one big number is being summed with many small numbers. 
For example, in four-digit decade floating-point space,
if 1000 small numbers 0.001 is accumulated to one large 
number 1.000, result would be 2.000. On the other hand,
if one large number 1.000 is summed with 1000 small numbers 0.001,
result would be 1.000 instead of 2.000 because of rounding errors.
\par
Those performances were measured, the accuracy was verified and this 
solution has been compared with other algorithms which is 
shown in further captions. This paper describes novel accurate 
summation algorithm utilizing fixed-point accumulator.
Algorithm is implemented in C++.


\section{Related work}
Several authors have used variety of methods for floating-point 
summation to find out which methods achieve the best accuracy. 
\par
The first algorithm solving accurate summation problem was Kahan's method \cite{Kahan}. 
Kahan's compensated summation method employs the correction 
on every step of summation. 
On every summation, correction for rounding error is computed and saved.
Correction is added to term in next iteration cycle.
If accumulator is large and terms are smaller, rounding error will be accumulated by correction variable, 
and would not be lost in ordinary summing.
Problem with Kahan is that correction will be lost if term in next iteration is relatively large to correction.
Because that usually practice is to sort array of terms before summing, 
which is inflexible because needs whole array of terms before summing.
\par
The Kahan-Babuska-Klein summation algorithm \cite{KahanBabuskaKlein} tries to overcome shortcomings of Kahan's by recursive correction.
This algorithm needs whole array of terms and also needs large amount of memory to save data between recursions.
\par
There are many others algorithms like \cite{DemmelAndHida} and \cite{DistillationAlgo} with comprehensive techniques 
to solving problem of accurate summation.
\par
In \cite{ComparisonOfMethods} many algorithms are compared with multiple problems.
The best of them is the method of Cascading Accumulators \cite{CascadingAccumulators}.
It uses array of 64 double precision floating-point accumulators.
Every single precision term with 4 consecutive exponent values is accumulated to appropriate cascading accumulator.
The maximal roundoff error given by this scheme is up to 4 bits.
Algorithm described in this paper has been compared to the method of Cascading Accumulators,
because the method of Cascading Accumulators is proved 
to be the best in performance and precision in \cite{ComparisonOfMethods}.
The fixed-point accumulator algorithm described in this paper practically has no rounding errors,
as well as the method of Cascading Accumulators.

\section{Implementation}
The fixed-point accumulator is implemented as C++ number class.
It has implemented typical methods for number class: 
arithmetic operators, assign operators, conversion operators, printing operators and support of manipulators.
The inline methods are used to give compiler the opportunity to make a good optimization and to fit variable into registers.
\par
So far, only single precision fixed-point accumulator is implemented. 
However, double precision could be implement with little additional effort.
\par
The accumulator is 320-bit multiple precision integer, which is implemented as array of 5 64-bit integers.
The compiler puts these numbers into registers, 
so the numbers are not stored into cache memory 
and therefore this solution is potentially faster than another ones.
The 0th bit of the mantissa of de-normalized number (exponent field equals 0) is mapped to the 0th bit of accumulator.
The 23rd bit of the accumulator represents 23rd bit of mantissa (hidden 1) with exponent field equals 1.
The 275th bit of the accumulator (20th bit of the 4th 64-bit integer) 
is mapped to the highest bit in the single precision floating-point (hidden 1 and exponent field 254).
Figure~\ref{fig:fix_acc_nums} shows look of fixed-point accumulator, with value representing floating-point number 1.0.
\par
Rest of bits, from 276th bit to 319th, are useful for expanding precision.
But when accumulator is converted back to single precision floating-point and any of those bits are set,
an overflow in fixed-point accumulator occurs.
Every number larger than 276-bit in fixed-point accumulator is cannot be represented with single precision floating-point.


\def\regWidth{64mm}
\def\regHeight{5mm}
\begin{figure}[!t]
	\centering
    \begin{tikzpicture}
		\node[anchor=east] at (-\regWidth, 8.5*\regHeight) {$acc_4$};
		\draw (0mm, 8*\regHeight) rectangle (-\regWidth, 9*\regHeight) node[pos=.5] {All 0s};
		\node[anchor=north] at (-\regWidth, 8*\regHeight) {319};
		\node[anchor=north] at (0mm, 8*\regHeight) {256};
		
		\node[anchor=east] at (-\regWidth, 6.5*\regHeight) {$acc_3$};
		\draw (0mm, 6*\regHeight) rectangle (-\regWidth, 7*\regHeight) node[pos=.5] {All 0s};
		\node[anchor=north] at (-\regWidth, 6*\regHeight) {255};
		\node[anchor=north] at (0mm, 6*\regHeight) {192};
		
		\node[anchor=east] at (-\regWidth, 4.5*\regHeight) {$acc_2$};
		\draw (-0.33*\regWidth, 4*\regHeight) rectangle (-\regWidth, 5*\regHeight) node[pos=.5] {0s};
		\draw (-0.3*\regWidth, 4*\regHeight) rectangle (-0.33*\regWidth, 5*\regHeight) node[pos=.5] {1};
		\draw (0mm, 4*\regHeight) rectangle (-0.3*\regWidth, 5*\regHeight) node[pos=.5] {0s};
		\node[anchor=north] at (-\regWidth, 4*\regHeight) {191};
		\node[anchor=north] at (-0.315*\regWidth, 4*\regHeight) {149};
		\node[anchor=north] at (0mm, 4*\regHeight) {128};
		
		\node[anchor=east] at (-\regWidth, 2.5*\regHeight) {$acc_1$};
		\draw (0mm, 2*\regHeight) rectangle (-\regWidth, 3*\regHeight) node[pos=.5] {All 0s};
		\node[anchor=north] at (-\regWidth, 2*\regHeight) {127};
		\node[anchor=north] at (0mm, 2*\regHeight) {64};
		
		\node[anchor=east] at (-\regWidth, 0.5*\regHeight) {$acc_0$};
		\draw (0mm, 0*\regHeight) rectangle (-\regWidth, 1*\regHeight) node[pos=.5] {All 0s};
		\node[anchor=north] at (-\regWidth, 0*\regHeight) {63};
		\node[anchor=north] at (0mm, 0*\regHeight) {0};
				
	\end{tikzpicture}
	
	\caption{Fixed-point accumulator having value $1.0$}
	\label{fig:fix_acc_nums}
\end{figure}


% Define block styles
\tikzstyle{decision} = [
	diamond, draw, fill=blue!20, 
	text width=4.5em, text badly centered, node distance=3cm, inner sep=0pt
]
\tikzstyle{block} = [
	rectangle, draw, fill=blue!20, 
	text width=5em, text centered, rounded corners, minimum height=4em
]
\tikzstyle{line} = [draw, -latex']
\tikzstyle{cloud} = [
	draw, ellipse,fill=red!20, node distance=3cm,
    minimum height=2em
]
\begin{figure}[!t]
	\centering
	\begin{tikzpicture}[node distance = 2cm, auto]
		% Place nodes
		\node [cloud] (input) {Floating-point number};
		\node [decision, below of=input] (is_denorm) {Is de-normal?};
		
		\node [block, left of=is_denorm, node distance=3cm] (denorm_add) {Accumulate mantissa to $acc_0$};
		\node [block, below of=denorm_add] (denorm_carry) {Propagate carry from $acc_0$ to $acc_4$};
		
		\node [block, right of=is_denorm, node distance=3cm] (exponent) {exponent to shift};
		\node [block, below of=exponent] (split) {Split shift};
		\node [block, below of=split] (local_shift) {Local shift of mantissa};
		\node [block, below of=local_shift] (acc) {Accumulate to $acc_{n+1}:acc_{n}$};
		\node [block, below of=acc] (carry) {Propagate carry from $acc_{n}$ to $acc_4$};
		\node [cloud, left of=carry, node distance=3cm] (end) {End};
		
		% Draw edges
		\path [line] (input) -- (is_denorm);
		
		\path [line] (is_denorm) -- node [near start] {yes} (denorm_add);
		\path [line] (denorm_add) -- (denorm_carry);
		\path [line] (denorm_carry) |- (end);

		\path [line] (is_denorm) -- node [near start] {no} (exponent);		
		\path [line] (exponent) -- (split);
		\path [line] (split) -- (local_shift);
		\path [line] (local_shift) -- (acc);
		\path [line] (acc) -- (carry);
		\path [line] (carry) -- (end);

	\end{tikzpicture}
	
	\caption{Algorithm of fixed-point accumulation}
	\label{fig:fix_acc_algo}
\end{figure}


\par
Figure~\ref{fig:fix_acc_algo} shows algorithm of direct accumulation of floating-points numbers to fixed-point accumulator.
Floating-points numbers which are accumulated do not need to be promoted to whole fixed-point type 
and then to be added on fixed-point accumulator.
Instead optimized approach is used.
If floating-point number is de-normalized its mantissa without hidden 1 is accumulated to lowest part of accumulator $acc_0$, 
without any shifting.
Depending is it positive or negative floating-point number, mantissa is added or subtracted to $acc_0$, respectively.
After that carry is propagated from $acc_0$ to $acc_4$.
Normalized numbers have to be shifted before accumulation, depending on their exponent. 
Shift value is calculated as exponent reduced by 1 and it is 8-bit wide.
Shift value is then splitted to 6 lower bit local shift and 2 higher bits for global shift.
Local shift value is used to shift mantissa with hidden 1 across two temporal 64-bit registers, 
lower $tmp_0$ and higher $tmp_1$.
If, for example, local shift is below 42, all mantissa bits will stay in lower $tmp_0$, 
otherwise some will be shifted to higher register $tmp_1$.
After local shifting, global shift decides to which accumulator number aforementioned two temporal registers are accumulated.
If global shift is $n$, $tmp_0$ is accumulated to $acc_n$
and $tmp_1$ is accumulated to $acc_{n+1}$.
Same as for de-normals, adding or subtraction is used for accumulation depending of floating-point number sign.
Carry is propagated $acc_n$, $acc_{n+1}$ and up to $acc_4$.
On this way, for large floating-point numbers,
unnecessary large shifting across multiple 64-bit registers and accumulation of zeros to $acc_{n-1}$ and below.
\par
In Figure~\ref{fig:fix_acc_example} example of accumulating $13194139533312.0$ is shown. 
This number have exponent field value set to 170 and its mantissa is 0 besides highest bit.
Shift value is 169, local shift is 41 and global is 2.
When mantissa with hidden 1 is shifted for 41, highest bit from mantissa will be highest bit in $tmp_0$
and hidden 1 will be lowest bit in $tmp_1$.
Then $tmp_0$ is added to $acc_2$ and $tmp_1$ to $acc_3$.
Carry which occurs when $tmp_1$ and $acc_3$ are added propagate to $acc_4$.

\def\regHeight{5mm}
\begin{figure}[!t]
	\centering
    \begin{tikzpicture}

		\node[anchor=west] at (-64mm, -0.5*\regHeight) {mantissa with hidden 1};
		\draw (-22mm, 0mm) rectangle (-24mm, -\regHeight) node[pos=.5] {\color{red} 1};
		\draw (-20mm, 0mm) rectangle (-22mm, -\regHeight) node[pos=.5] {\color{blue} 1};
		\draw (0mm, 0mm) rectangle (-20mm, -\regHeight) node[pos=.5] {22x 0s};
		\node[anchor=north] at (-23mm, -\regHeight) {23};
		\node[anchor=north] at (0mm, -\regHeight) {0};

		\node[anchor=west] at (-70mm, -2.5*\regHeight) {Local shifting};

		\node[anchor=east] at (-64mm, -4.5*\regHeight) {$tmp_1$};
		\draw (-2mm, -4*\regHeight) rectangle (-64mm, -5*\regHeight) node[pos=.5] {0s};
		\draw (0mm, -4*\regHeight) rectangle (-2mm, -5*\regHeight) node[pos=.5] {\color{red} 1};
		\node[anchor=north] at (-64mm, -5*\regHeight) {127};
		\node[anchor=north] at (0mm, -5*\regHeight) {64};

		\node[anchor=east] at (-64mm, -6.5*\regHeight) {$tmp_0$};
		\draw (-62mm, -6*\regHeight) rectangle (-64mm, -7*\regHeight) node[pos=.5] {\color{blue} 1};
		\draw (0mm, -6*\regHeight) rectangle (-62mm, -7*\regHeight) node[pos=.5] {0s};
		\node[anchor=north] at (-64mm, -7*\regHeight) {63};
		\node[anchor=north] at (0mm, -7*\regHeight) {0};
				

		\node[anchor=east] at (-64mm, -10.5*\regHeight) {$acc_4$};
		\draw (0mm, -10*\regHeight) rectangle (-64mm, -11*\regHeight) node[pos=.5] {All 0s};
		\node[anchor=north] at (-64mm, -11*\regHeight) {319};
		\node[anchor=north] at (0mm, -11*\regHeight) {256};		
		
		\node[anchor=east] at (-64mm, -12.5*\regHeight) {$acc_3$};
		\draw (0mm, -12*\regHeight) rectangle (-64mm, -13*\regHeight) node[pos=.5] {All 1s};
		\node[anchor=north] at (-64mm, -13*\regHeight) {255};
		\node[anchor=north] at (0mm, -13*\regHeight) {192};
		
		\node[anchor=east] at (-64mm, -14.5*\regHeight) {$acc_2$};
		\draw (0mm, -14*\regHeight) rectangle (-64mm, -15*\regHeight) node[pos=.5] {All 0s};
		\node[anchor=north] at (-64mm, -15*\regHeight) {191};
		\node[anchor=north] at (0mm, -15*\regHeight) {128};
		
		\node[anchor=west] at (-70mm, -16.5*\regHeight) {Accumulation and carry propagation};

		\node[anchor=east] at (-64mm, -18.5*\regHeight) {$acc_4$};
		\draw (-2mm, -18*\regHeight) rectangle (-64mm, -19*\regHeight) node[pos=.5] {63x 0s};
		\draw (0mm, -18*\regHeight) rectangle (-2mm, -19*\regHeight) node[pos=.5] {1};
		\node[anchor=north] at (-64mm, -19*\regHeight) {319};
		\node[anchor=north] at (0mm, -19*\regHeight) {256};

		\node[anchor=east] at (-64mm, -20.5*\regHeight) {$acc_3$};
		\draw (0mm, -20*\regHeight) rectangle (-64mm, -21*\regHeight) node[pos=.5] {All 0s};
		\node[anchor=north] at (-64mm, -21*\regHeight) {255};
		\node[anchor=north] at (0mm, -21*\regHeight) {192};
		
		\node[anchor=east] at (-64mm, -22.5*\regHeight) {$acc_2$};
		\draw (-62mm, -22*\regHeight) rectangle (-64mm, -23*\regHeight) node[pos=.5] {\color{blue} 1};
		\draw (0mm, -22*\regHeight) rectangle (-62mm, -23*\regHeight) node[pos=.5] {63x 0s};
		\node[anchor=north] at (-64mm, -23*\regHeight) {191};
		\node[anchor=north] at (0mm, -23*\regHeight) {128};		
		
	\end{tikzpicture}
	
	\caption{Example of accumulating floating-point number $13194139533312.0$}
	\label{fig:fix_acc_example}
\end{figure}

\par
Advantage of the solution shown in this paper in comparison to the method of Cascade Accumulator 
is that it has up to 12 times less memory used.
Fixed-point accumulator need 40 bytes, 
comparing with 512 Cascade Accumulator implemented in the \cite{ComparisonOfMethods}.
\par
Source code of implementation described here could be found in \cite{FixAccGitHub}.


\section{Measurements}
This paper presents three problems: 'Many smalls, one large',
'One large, many smalls' and 'One large, many smalls, many tinies'.
In the following, fixed-point algorithm is compared with Kahan's method
which is classic algorithm and the method of Cascading Accumulators 
which presents the best solution due to its performance and precision 
proved in paper \cite{CascadingAccumulators}.

\par
'Many smalls, one large' is problem where many small numbers 
are being summed with one large number. 
Large number is 1.0 and smaller number is $2^{24} \epsilon$.
In this paper $\epsilon$ is unit roundoff relative to 1.0 and 
is $b^{-(p-1)}/2$ or concretely for single precision float point 
$2^{-24}$ as defined by Demmel \cite{DemmelsEpsilon}.
Small numbers are accumulated to one large number 1.0, the result would be 2.0.
\par
'One large, many smalls' is problem where one number 
is being summed with many smaller numbers. 'One large, many smalls' 
test consist of adding one number 1.0 and $2^{24}$ number $\epsilon$, 
which is relatively small in compare with number 1.0.
Тhe result would be 1.0 instead of 2.0 because of rounding errors.
\par
'One large, many smalls, many tinies' is problem where one 
number is being summed with $2^{24}$ small numbers and $2^{26}$
tiny numbers. $2^{26}$ tiny numbers is equivalent to 4 small numbers.
'One large, many smalls, many tinies' test consist of adding 
one number 1.0 and $2^{24}$ number $\epsilon$ and $2^{26}$ number 
$epsilon^2$.
\par
The various methods and data were tested by running C++ programs.
The results are shown in Table~\ref{tab:measurement}. 
Two numbers are shown for each method and problem, 
one that represents obtained value of the algorithm,
and another that represents the time required for execution of problem in 100 iteration.

\begin{table}[!t]
\renewcommand{\arraystretch}{1.3}
\caption{Result and measured time for 100 iteration}
\label{tab:measurement}
\begin{tabular}{|c|c|c|c|}
\hline
	\parbox[]{1.7cm}{ \centering Methods } & 
	\multicolumn{3}{|p{5.1cm}|}{\centering Test problems } \\
\hline
	\parbox[]{1.7cm}{ \center } & 
	\parbox[]{1.7cm}{ \centering Many smalls, one large } & 
	\parbox[]{1.7cm}{ \center One large, many smalls } & 
	\parbox[]{1.7cm}{ \center One large, many smalls, many tinies \linebreak } \\
\hline
	Correct result &
	\parbox[]{1.7cm}{ \center 2.0 \linebreak} &
	\parbox[]{1.7cm}{ \center 2.0 \linebreak} &
	\parbox[]{1.7cm}{ \center $2.0 + 4\epsilon$ \linebreak} \\
\hline
	Ordinary sum &
	\parbox[]{1.7cm}{ \center 2.0 \linebreak 7.194s \linebreak} &
	\parbox[]{1.7cm}{ \center 1.0 \linebreak 7.258s \linebreak} &
	\parbox[]{1.7cm}{ \center 1.0 \linebreak 34.30s \linebreak} \\
\hline
	Kahan &
	\parbox[]{1.7cm}{ \center 2.0 \linebreak 12.70s \linebreak} &
	\parbox[]{1.7cm}{ \center 2.0 \linebreak 12.69s \linebreak} &
	\parbox[]{1.7cm}{ \center 2.0 \linebreak 62.32s \linebreak} \\
\hline
	\parbox[]{1.7cm}{ \center Cascade \linebreak accumulator \linebreak} &
	\parbox[]{1.7cm}{ \center 2.0 \linebreak 7.008s \linebreak} &
	\parbox[]{1.7cm}{ \center 2.0 \linebreak 6.977s \linebreak} &
	\parbox[]{1.7cm}{ \center $2.0 + 4\epsilon$ \linebreak 34.93s \linebreak} \\
\hline
	\parbox[]{1.7cm}{ \center Fixed-point \linebreak accumulator \linebreak} &
	\parbox[]{1.7cm}{ \center 2.0 \linebreak 11.69s \linebreak} &
	\parbox[]{1.7cm}{ \center 2.0 \linebreak 11.70s \linebreak} &
	\parbox[]{1.7cm}{ \center $2.0 + 4\epsilon$ \linebreak 58.53s \linebreak} \\	
\hline
\end{tabular}
\end{table}

\par
Kahan's algorithm does not work for problem called 'One large, many smalls, many tinies'.
Correction variable is larger or equal than small number in the moment when tinies numbers are started to be accumulated,
and tinies are lost because there are order of $\epsilon$ of small number.
For example, according to previous explanation of problem 'One large, many smalls, many tines', 
it is expected to get a result $2.0 + 4\epsilon$, but the result is 2.0, 
because $4\epsilon$ in tinies is lost.
To solve this problem Kahan's algorithm need compensation for its compensation variable.
This imply usage of recursive algorithms like Kahan-Babuska-Klein.

\par
Both fixed-point accumulator and Cascading Accumulators method are without rounding errors.
In case of Cascading Accumulators method it is possible to have overflow on particular accumulator.
Particular implementation of Cascading Accumulators used in this work have approximately 27 bits before overflow.
Fixed-point accumulator have approximately 41 bits before overflow.
Making problem to test overflow robustness of Cascading Accumulators method and fixed-point accumulator solution will be future work.
\par
The fixed-point accumulator is comparable with Cascading Accumulators in performances.
Solution described here is less than two times slower than Cascading Accumulators.
Reason for that is that those 64 
double accumulators are in cache memory and fixed-point accumulator 
has a bit more work that is a bit larger number of instructions 
per summing than the Cascading Accumulators, so it comes to the same.
Even with slightly worse performance, fixed-point accumulator need
more than 12 times less memory in compare to Cascading Accumulators.
\par
Bad side of Cascading Accumulators is that it needs double precision floating-point accumulators.
These accumulators are not always available. Due to this facts, implementation of 
Cascading Accumulators for double precision could not be possible.
The fixed-point accumulator could be implemented on any architecture.
\par
The fixed-point accumulator would be better in hardware due to the fact that it 
occupies much less logic and it is faster than the floating-point units.

\section{Conclusion}
In this paper it has been proven
that the accurate summation using fixed-point accumulator 
algorithm do not have rounding errors as well as the method of Cascading Accumulators,
the best known algorithm.
It is shown that solution presented in this paper has comparable performances as Cascading Accumulators method.
\par
Further improvement in C++ can be achieved by optimizing 
solution in assembler and implementing double precision version.
Also hardware implementation should be implemented and compared to other algorithms.
This solution should also be compared to more algorithms and problems. 

% use section* for acknowledgment
\section*{Acknowledgment}
This work was partially supported by the Ministry of Education, 
Science and Technological Development of the Republic of Serbia, 
under grant number: TR32029.


\bibliographystyle{IEEEtran}
\bibliography{cites}


\end{document}
